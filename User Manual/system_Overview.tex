\subsection{Overview Description:}
	The COSBAS (Computer Science Biometric Access System) is a secure system that uses Bio-metric inputs (such as facial recognition and fingerprint scanning) to unlock and gain access to the department and offices.
	
\subsection{Pinpoint Descriptions:}
	\subsubsection{COSBAS-Client}
	The COSBAS client is the hardware aspect of the COSBAS system, and should deployed on  Raspberry Pi's at the entrances and exits of the departement.  It captures biometric data or access codes from the users and sends it to the server for authentication. On successful authentication the door is opened and the user is let into the department. 
	
	Currently the COSBAS system supports authentication only through access codes, Facial Recognition and Fingerprint Identification, but the system is highly pluggable so more  authentication methods can be added later. 
	
	\textbf{Facial recognition}
	The client takes a photograph that is sent to the server. On the server the image is grayscaled and the most center face extracted for authentication. 
	
	\textbf{Fingerprint Recognition}
	Fingerprint images are close proximity images and hence will always be accurate enough to send it directly to the server for authentication.
	
	\subsubsection{Web-Client}
	The interface for the COSBAS system is a responsive webpage the user (authorized and temporary visitors) may use to request permission for access to the department. They can also book appointments with members of faculty on the system.
	
	\subsubsection{Bookings and Appointments}
	An unauthorized user can book an appointment with an employee enrolled in the system by making use of a Calendar integration feature on the web based interface. Authorized users can then either accept or decline the booking for an appointment of which the person whom made the booking is notified of the status of their booking via an email.
	
	\subsubsection{Temporary Access}
	Once a booking has been accepted by the authorized user of COSBAS, the relevant users will be notified via email containing a link that will expose their temporary access code generated by the COSBAS system. A link is added in the email that the user can click on to cancel the appointment with the associated COSBAS user. In such a case, the temporary access code will be revoked.