% %
\subsubsection{Human access channels}
This system will be accessible to humans in the followings ways:

	\begin{itemize}
		\item From a thin client (this can be any computer with the client program but the preference is a Raspberry Pi) which will be installed at each entrance/exit of the building through non-intrusive bio-metrics or keypad.
		\item Humans can access the web client in the following ways:
		\begin{itemize}
			\item Web Browser (client)
				\begin{itemize}
					\item A web browser is needed to display the web pages of the website
					\item Specifically Firefox, Chrome, Opera, Safari and IE.
				\end{itemize}
			\item Physical Devices
			\begin{itemize}
				\item Since the website will be responsive, any device which has a web browser will be able to connect to the website
				\item PCs, Laptops, Tablets and Smartphones
			\end{itemize}
		\end{itemize}
	\end{itemize}
	
\subsubsection{System access channels}
The client(can be computer with the client program but in this case it will be a Raspberry Pi) should be able to access the services provided by the system to authenticate a user who would like to enter or exit the building. This will be done through SOAP based web services.

In order to access the web client, the following hardware will be needed:
\begin{itemize}
	\item Internet Connection (ADSL) (wired or Wi-Fi)
	\begin{itemize}
		\item A decent internet connection (at least ADSL level speed – 384kbs) will be needed to connect to the website (since its being run on a server)
	\end{itemize}
	\item Ethernet Connection
	\begin{itemize}
		\item This is necessary if an Internet Connection is not available – specifically if the web server is run locally, local computers can connect to it via Ethernet and thus have no need to use an Internet Connection
	\end{itemize}
\end{itemize}

