\subsection{Software}

\subsubsection{Build System}
This project will use the Gradle build system.

\subsubsection{Server}
\textit{Programming Language:} Java \\
\textit{Platform:} The server-program needs to be deployed on a Linux server.\\
\textit{Application Server:} Jetty.\\
\textit{Reasoning: } We chose a Jetty server over other application servers such as TomCat because the Jetty application server is much more third-party API friendly. Jetty is also much more focused on scalability and better performance with regards to serving static content than TomCat. Jetty's focus on multi-connection HTTP and features such as SPDY can significantly reduce page load latencies. The ultimate focus of Jetty is on speed, scalability and reliability, which is very important for our system.

\subsubsection{Frameworks and libraries:}
\begin{itemize}
	\item Spring
		\begin{itemize}
			\item Spring Framework (MVC, IoC, AOP)
				\begin{itemize}
					\item We chose the Spring framework over other technologies for the following reasons:
					\item Dependencies are explicit and evident in JavaBean properties.
					\item Spring enhances modularity.
					\item Spring provides more readable code.
					\item Spring allows us to have loose coupling between different modules. This means that we can inject dependencies at runtime.
					\item Spring provides a more flexible way to do dependency injection - dependencies can be configured by XML based schemas or annotations.
					\item Spring makes it easier to implement Inversion of Control because we leave the work of dependency injection to the underlying framework.
					\item Spring AOP (aspect oriented programming) is implemented purely in Java, and thus we don't need a separate compilation process. Also, because Spring AOP integrates with the underlying Spring framework, we get an advantage in terms of declarative programming for our security and logging purposes.
					\item Spring isn't application server dependent.
					\item Spring doesn't require any special deployment steps.
					\item Spring simplifies Unit Testing because of its loose coupling, thus it makes it very easy to test a class independently (with or without mock objects).
					\item Spring is innovative - for example, Spring was the first to bring out CDI, while other technologies like JavaEE took nearly 3 years to do the same thing. Thus, the possibility exists that Spring will keep innovating their technology and bring out new features. For example, Spring implements Spring Data, Spring Social and Spring Mobile.
					\item Spring enables POJO programming which enables continuous integration and testability.
					\item Spring is open source and has no vendor lock in.
					\item Spring has a layered architecture, which means we only have to use what we need and we can leave what we don't.
					\item The main reason why we are using it, however, is because of it's outstanding MVC framework. It is highly configurable with strategy interfaces, which is one of the requirements of our project (because we need to be able to use different types of Biometric Access Systems).
				\end{itemize}
			\item Spring LDAP
			\item Spring Data MongoDB
				\begin{itemize}
					\item Enabled easy integration with MongoDB in Java.
					\item Provides build in operations/functions for the CRUD operations of the database.
					\item Has CDI support for the Mongo Repositories that enables the system to have custom query functions.
				\end{itemize}
			\item Spring Security
		\end{itemize}
	\item Thymeleaf template language for HTML and XML
	\item OpenCV graphics processing library
	\item JasperReports
\end{itemize}

\subsubsection{Database}
COSBAS will have a MongoDB database for persistence of the Biometric Data of the staff members as well as the appointment information of each staff member and the authentication keys that is generated for each appointment. \\
\\
The reasons for using MongoDB:
	\begin{itemize}
		\item The flexible data model allows us to persist our images and sounds that will be used as the biometric data to authenticate a staff member.
		\item The scalability that MongoDB provides enable us to reach our scalability quality requirement (Horizontal Scaling) by the means of splitting the database up to run over two servers if we need to accomodate more staff members or visitors in the future.
	\end{itemize}

\subsubsection{Client}
The client application will be developed in either Python or Java depending on the biometric devices' APIs. It will communicate with the server using http requests. Each http request's data will contain the client's door id, whether the user is entering or exiting and the types of biometric devices and the captured data. 

The client application should run on Linux (Raspbian)

\subsubsection{Web Access Channel}
\begin{itemize}
	\item HTML5
	\item Javascript
	\item JQuery
	\item Ajax
	\item Bootstrap
\end{itemize}

\subsubsection{Testing}
We will use JUnit for unit testing.