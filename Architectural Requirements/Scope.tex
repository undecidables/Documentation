\subsubsection{Persistence}
Persisting system valuable information will take place in a database environment. In the scope of this project a non-relational and NoSQL database will be used. MongoDB was chosen as the best solution for the COSBAS project. The reason for this is the schemaless structure of data so there is no need to keep to a strict schema and MongoDB also allows us to store image data for biometrics.

\subsubsection{Communication}
A client-server architecture will be used for communication. We will use http requests sent from either the web browser or from the rasberry-pi computers. 

\subsection{Notifications}
The appointment system should be able to send notifications to lecturers and visitors. All notifications will be via email, but the system should stay pluggable with the possibility of adding other methods later. In this context, methods refer to sub-systems such as biometric authentication services, third-party calendar services and notification method services.

\subsection{Biometric Validation}
The system should be able to validate biometric data sent from the client to identify the user it belongs to and grant access to the department for authorized users.