\subsection{Requesting Appointment}
\textit{Priority: \textcolor{orange}{Important}} 

\subsubsection{ Description}
The requestAppointment function allows a user of the system to request an appointment with a staff member that is also making use of the system. 

\subsubsection{Pre/Post conditions}
\textbf{Pre-conditions:} 
	\begin{itemize}
		\item Lecturer must exist.
		\item Date and time of the requested appointment must be valid entries.
	\end{itemize}
\textbf{Post-conditions:} 
	\begin{itemize}
		\item Appointment will be saved for the lecturer to approve or deny later on.
		\item User gets an appointment identifier back.
		\item Lecturer is notified of the requested appointment. 
	\end{itemize}

\subsection{Cancelling Appointment}
\textit{Priority: \textcolor{orange}{Important}} 

\subsubsection{Description}
The cancelAppointment function allows the user to cancel an appointment if the user is either the staff member who's appointment it is or the user who made the appointment. 

\subsubsection{Pre/Post conditions}
\textbf{Pre-conditions:} 
	\begin{itemize}
		\item The appointment must exist.
		\item The user cancelling the appointment has to be the person that the appointment is with or the person who made.
	\end{itemize}
\textbf{Post-conditions:} 
	\begin{itemize}
		\item The appointment will be cancelled.
		\item  Both parties are notified.
		\item Access that was granted for the appointment is revoked. 
	\end{itemize}

\subsection{Use Case diagram for Requesting Appointment and Cancelling Appointment}
	\includegraphics[width=\linewidth]{AlterAppointment/alterAppointmentUseCase.jpg}
	
\subsection{Functionality for Requesting Appointment and Cancelling Appointment}
	\includegraphics[width=\linewidth]{AlterAppointment/alterAppointmentRequiredFunctionality.jpg}
	
\subsection{Process Specification for Requesting Appointment and Cancelling Appointment}
	\includegraphics[width=\linewidth]{AlterAppointment/alterAppointmentProcess.jpg}

