The following functionality of creating and setting availability will only be available to staff members. Users who are not authorised by the system will be granted read-only privileges.
\subsection{Use Case Diagram}
\includegraphics[width=\linewidth]{Availability/Setting_Of_Availability1.jpg}

\subsection{Module Functionality}
\includegraphics[width=\linewidth]{Availability/Setting_Of_Availability.jpg}


\subsection{Choose Calendar}
\textit{Priority: \textcolor{orange}{Important}}
	\subsubsection{Description}
	An authorised user of the system should be able to select which calendar account to associate with the system. If the user chose a linked service, then that service needs to be authenticated.\\
	\subsubsection{Pre/Post conditions}
		\textbf{Pre-conditions:} The user should be logged into the system and already have a calendar account from the linked services (Google \& Outlook). \\
		\textbf{Post-conditions:} Once the user has chosen the calendar to associate with the system, all updates in availability will be amended in that specific calendar.

\subsection{View Availability}
\textit{Priority: \textcolor{orange}{Important}}
	\subsubsection{Description}
	Both the authorised user and a non-authorised user (or guest) should be able to have a read-only functionality of viewing the availability of the associated staff member on their calender.\\
	\subsubsection{Pre/Post conditions}
		\textbf{Pre-conditions:} An associated calendar for the staff member should exist.\\
		\textbf{Post-conditions:} Be able to see whether the associated staff member has an appointment at a certain date and time or if they are free for appointments. 
	
\subsection{Set Availability}
\textit{Priority: \textcolor{orange}{Important}}
	\subsubsection{Description}
	The authorised user should be able to set whether they are available at a certain date or time or if they have an appointment at a specific date or time. This can be set by the staff member himself/herself or by the approval of an appointment.\\
	\subsubsection{Pre/Post conditions}
		\textbf{Pre-conditions:} 
		\begin{itemize}
			\item The user should be logged in.
		 	\item A calendar should exist.
		\end{itemize}
		\textbf{Post-conditions:} The linked calendar should be updated with the particulars specified by the authorised user. (CRUD of appointments).