\documentclass[11pt,a4paper,titlepage]{article}

\usepackage{pdflscape}
\usepackage[margin=1in]{geometry}
\usepackage{titling}
\usepackage{graphicx}
\usepackage[hidelinks]{hyperref}
\usepackage{lscape}

\graphicspath{ {./images/} }


\usepackage{color}

\definecolor{orange}{RGB}{255,130,0}
\definecolor{myGreen}{RGB}{0,128,0}


\newcommand{\subtitle}[1]{
  \posttitle{
    \par\end{center}
    \begin{center}\large#1\end{center}
    \vskip0.5em}
}

\begin{document}
\title{COSBAS Functional Requirements Documentation}
\subtitle{ Git: \url{https://github.com/undecidables/Requirements-Documentation} \\ GitHub Organisation: \url{https://github.com/undecidables}}
\begin{figure}
			\centering
			\includegraphics[width=400px]{GroupLogo.png}
\end{figure}
\author{
\textbf{The Client:}\\
Prof Andries Engelbrecht \\
Head of Department, Computer Science\\
University Of Pretoria
\\
\\
\textbf{The Team:}\\
			Elzahn Botha  \emph{13033922} \\
			Jason Richard Evans \emph{13032608} \\
			Renette Ros \emph{13007557} \\
			Szymon Ziolkowski \emph{12007367} \\
			Tienie Pritchard  \emph{12056741} \\
			Vivian Venter \emph{13238435} \\
}   
\date{\textbf{March 2015}}

\maketitle

\tableofcontents
\pagebreak

\setlength{\parindent}{0em}
\setlength{\parskip}{0.5em}

%I strongly recomend using seperate tex files for each of the following sections and just include them here.

	\section{Introduction}
	In this section of the document we will identify and address a high level overview of the COSBAS (Computer Science Biometric Access System) system. The reason for building this system is because currently it is too difficult for students to gain authorized access to the department when they have a meeting with a lecturer. There is also no means of gathering data on who is entering or leaving the department and there is no easy means of creating or requesting appointments with a specific lecturer.
	
	\section{Vision and Objectives}
	\subsection{Vision}
	The vision of the COSBAS system is to create a biometric access control system that makes use of different biometric devices and the latest mobile technologies to grant access to visitors and staff members.The system will also serve as an appointment system between lecturers, visitors and students.

	\subsection{Objectives}
	The main objectives of the COSBAS system is:
	\begin{itemize}	
 		\item to provide secure biometric access to not only staff members of the department, but visitors as well.
		\item to provide the ability to make appointments by means of a web interface. 
	\end{itemize}

	\subsection{Terminology/Clarifications}
	\begin{itemize}	
 		\item A \textbf{booking} is when a user, either a guest or a staff member, requests an appointment with a staff member. 
		\item An \textbf{appointment} is when the particular staff member, with whom the booking is made, approves the booking.
	\end{itemize}

\begin{landscape}
	\section{Domain Model}	
	\includegraphics[width=\linewidth]{COSBAS_Domain}
	\end{landscape}
	
	%if we really want to we can make all these one level lower and include a modules heading...
		%use cases with priority, functionality, process in here - see image in planning folder/giiter chat
		
	\section{Appointments}
	%Vivian
\subsection{Approve Appointment}
\textit{Priority: \textcolor{orange}{Important}} 

\subsubsection{Description}
A person can book for an appointment with one of the staff members in the department/building. The particular staff member needs to approve such an appointment. A booking needs to be approved first by a staff member before the guest/visitor can gain access to the department/building.

\subsubsection{Use Case diagram}
\includegraphics[width=\linewidth]{Appointments/ApproveAppointment}

\subsubsection{Pre-/Post-Conditions}
\textbf{Pre-conditions:} 
	\begin{itemize}
		\item Staff member needs to be logged in to approve his/her own appointments.
		\item There should be a booking before an appointment can be approved.
	\end{itemize}
\textbf{ Post-conditions:} 
	\begin{itemize}
		\item The staff member will be booked for that time period, therefor the staff member will be unavailble during that time.
		\item An email notification with an temporary access code is sended to the user whom booked for the appointment.
	\end{itemize}

%Vivian
\subsection{Disapprove Appointment}
\textit{Priority: \textcolor{orange}{Important}}

\subsubsection{Description}
An appointment can be disapproved by a staff member. The user who booked for the appointment will be notified that the appointment has been disapproved by the particular staff member. 

\subsubsection{Use Case diagram}
\includegraphics[width=\linewidth]{Appointments/DisapproveAppointment}

\subsubsection{Pre-/Post-Conditions}
\textbf{Pre-conditions:} 
	\begin{itemize}
		\item Staff member needs to be logged in to disapprove his/her own appointments.
		\item There should be a booking before an appointment can be disapproved.
	\end{itemize}
\textbf{ Post-conditions:} 
	\begin{itemize}
		\item An email notification informing the user whom made the booking that the appointment has been disapproved.
	\end{itemize}	

\subsection{Process Specification for Approve and Disapprove Appointment}
	\includegraphics[width=\linewidth]{Appointments/ApproveDisapproveAppointmentProcess}	
	\subsection{Requesting Appointment}
\textit{Priority: \textcolor{orange}{Important}} 

\subsubsection{Description}

This functionality allows a user of the system to request an appointment with a staff member that is also using system. 


\subsubsection{Pre-/Post-Conditions}
\textbf{Pre-conditions:} 
	\begin{itemize}
		\item Staff member must exist.
		\item Date and time of the requested appointment must be valid entries.
	\end{itemize}
\textbf{Post-conditions:} 
	\begin{itemize}
		\item Appointment will be saved for the staff member to approve or disapprove later on.
		\item User will receive an appointment identifier.
		\item Staff member is notified of the requested appointment. 
	\end{itemize}

\subsection{Cancelling Appointment}
\textit{Priority: \textcolor{orange}{Important}} 

\subsubsection{Description}
This function allows the user to cancel an appointment they have made, or in the case of a staff member, that has been made with them. The user who made the appointment will use the appointment identifier to cancel the booking. 


\subsubsection{Pre-/Post-Conditions}
\textbf{Pre-conditions:} 
	\begin{itemize}
		\item The appointment must exist.
		\item The user cancelling the appointment has to be the person that the appointment is with or the person who made it.
	\end{itemize}
\textbf{Post-conditions:} 
	\begin{itemize}
		\item The appointment will be cancelled.
		\item  Both parties are notified.
		\item Access that was granted for the appointment is revoked. 
	\end{itemize}

\subsection{Use Case diagram for Requesting Appointment and Cancelling Appointment}
	\includegraphics[width=\linewidth]{AlterAppointment/alterAppointmentUseCase.jpg}
	
\subsection{Functionality for Requesting Appointment and Cancelling Appointment}
	\includegraphics[width=\linewidth]{AlterAppointment/alterAppointmentRequiredFunctionality.jpg}
	
\subsection{Process Specification for Requesting Appointment and Cancelling Appointment}
	\includegraphics[width=\linewidth]{AlterAppointment/alterAppointmentProcess.jpg}


	
	\section{Availability}
	The following fucntionality of creating and setting will be available to only signed in users. Users who are not authorised or signed into the system will only be granted read-only privileges.

\includegraphics[width=\linewidth]{Availability/Setting_Of_Availability1.jpg}


\subsection[Choose Calendar]\hfill \textit{Priority: Critical} \\
	\subsubsection{Description}
	An authorised user of the system should be able to select which calendar account to associate with the system. If the user chose a linked service, then that service needs to be authenticated.\\
	\subsubsection{Pre/Post conditions}
		\textbf{Pre-conditions:} The user should be logged into the system and already have a calendar account from the linked services (Google \& Outlook). \\
		\textbf{Post-conditions:} Once the user has chosen the calendar to associate with the system, all updates in availability will be amended in that specific calendar.
	\subsubsection{Functionality}
	\subsubsection{Process Specification}

\subsection{View Availability}\hfill \textit{Priority: Critical} \\
	\subsubsection{Description}
	Both the authorised user and a non-authorised user (or guest) should be able to have a read-only functionality of viewing the availibility of the associated staff member on their calender.\\
	\subsubsection{Pre/Post conditions}
		\textbf{Pre-conditions:} An associated calendar for the staff member should exist.\\
		\textbf{Post-conditions:} Be able to see whether the associated staff member has an appointment at a certain date and time or if they are free for appointments. 
	\subsubsection{Functionality}
	\subsubsection{Process Specification}
	
\subsection{Set Availability}\hfill \textit{Priority: Critical} \\
	\subsubsection{Description}
	The authorised user should be able to set whether s/he is available at a certain date or time or if they have an appointment at a specific date or time. This can be set by the staff member himself/herself or by the approval of an appointment.\\
	\subsubsection{Pre/Post conditions}
		\textbf{Pre-conditions:} The user should be authorised and logged in and a calendar should exist.\\
		\textbf{Post-conditions:} The linked calendar should be updated with the particulars specified by the authorised user. (CRUD of appointments).
	\subsubsection{Functionality}
	\subsubsection{Process Specification}
		
	\section{Biometric Access}
	
This module encapsulates functionality regarding validating biometric data or temporary access code to give staff members or visitors access to the Computer Science Department

\includegraphics[width=\linewidth]{Access/BiometricDomain}

\subsection{Request Access}
\textit{ Priority: \textcolor{red}{Critical}} \\


\textbf{Staff Access} A staff member can gain access at a door using biometrics. \\
\textbf{Visitor Access} A visitor can enter the department by entering their temporary access code at the door.

\subsubsection{Use Case diagram}
\includegraphics[width=\linewidth]{Access/RequestAccess}

\subsubsection{Pre-/Post-Conditions}
\textbf{Staff Access Pre-conditions:} 
	\begin{itemize}
		\item Staff member must be registered on system.
		\item Biometric data must validate correctly
	\end{itemize}
\textbf{Visitor Access Pre-conditions:} 
	\begin{itemize}
		\item Visitor needs to have an appointment
		\item Temporary access code should be valid
	\end{itemize}
\textbf{ Post-conditions:} 
	\begin{itemize}
		\item User gains access
		\item Access logged
	\end{itemize}
	
\subsubsection{Process Specification}
	\includegraphics[width=\linewidth]{Access/RequestAccessActivity}	

%Vivian
\subsection{Register User}
\textit{Priority: Critical} \\

\subsubsection{Description}
To register a user on the system is the same as to store biomteric data of the user on the database.
Staff members needs to store their biomteric data to gain access to department/building. There will be an administrator(admin user) that will handle all the registering of users on the system.

\subsubsection{Use Case diagram}
\includegraphics[width=\linewidth]{Access/RegisterUser}

\subsubsection{Pre-/Post-Conditions}
\textbf{Pre-conditions:} 
	\begin{itemize}
		\item Administrator (admin user) needs to be logged in to register a user.
		\item The user being register should be authenticated using LDAP. 
		\item The user being registered must be a staff member.
	\end{itemize}
\textbf{ Post-conditions:} 
	\begin{itemize}
		\item The staff member is registered on the system with his/hers own biometric data.
		\item The staff member will be able to gain access to department/building.
	\end{itemize}
	
\subsubsection{Process Specification}
	%\includegraphics[width=\linewidth]{}	
	
	\section{Reporting}
	\includegraphics[width=\linewidth]{Reporting/Scope}
\subsection{Get Visitor Access Exit Times Report}
\textit{Priority: \textcolor{orange}{Important}}

\subsubsection{Description}
The getVisitorAccessExitTimes function allows a user to query the access and exit times of a specific visitor.

%	\textbf{Parameters:}
%		\begin{itemize}
%			\item VisitorId - specifies for which visitor to get the access and exit times.
%			\item SpecificDay - Optional parameter, specifies which day to get the access and exit times of a visitor.
%			\item FromDate and EndDate - Optional paramters, specifies which days to get the access and exit times of a visitor.
%		\end{itemize}
		
\subsubsection{Service Contract}
\includegraphics[width=\linewidth]{Reporting/getVisitorAccessExitTimesReport_serviceContract}
\subsubsection{Pre/Post Conditions}
	\textbf{Pre-conditions:}
	\begin{itemize}
		\item User must be logged in.
		\item User must have the correct authorization to make use of this function.
		\item Visitor must exist.
		\item Date must be valid.
	\end{itemize}

\subsection{Get Staff Access Exit Times Report}
\textit{Priority: \textcolor{orange}{Important}}

\subsubsection{Description}
The getStaffAccessExitTimes function allows a privileged user(e.g: Head of Department) to query the access and exit times of a specific staff member.

%	\textbf{Parameters:}
%	\begin{itemize}
%		\item StaffId - specifies for which staff member to get the access and exit times.
%		\item SpecificDay - Optional parameter, specifies which day to get the access and exit times of a staff member.
%		\item FromDate and EndDate - Optional paramters, specifies which days to get the access and exit times of a staff member.
%	\end{itemize}

\subsubsection{Service Contract}
\includegraphics[width=\linewidth]{Reporting/getStaffAccessExitTimesReport_serviceContract}
\subsubsection{Pre/Post Conditions}		
	\textbf{Pre-conditions:}
	\begin{itemize}
		\item User must be logged in.
		\item User must have the correct authorization to make use of this function.
		\item Date must be valid.
		\item Staff member must exist.
	\end{itemize}

\subsection{Get Not Honoured Appointments Report}
\textit{Priority: \textcolor{orange}{Important}}

\subsubsection{Description}
The getNotHonouredAppointments function provides the user with a means of querying a visitor's or staff member's not honoured appointments.

%	\textbf{Parameters:}
%	\begin{itemize}
%		\item StaffId or VisitorId - specifies for which staff member or visitor to get the not honoured appointments.
%	\end{itemize}
		
\subsubsection{Service Contract}
\includegraphics[width=\linewidth]{Reporting/getNotHonouredAppointmentReports_serviceContract}
\subsubsection{Pre/Post Conditions}
	\textbf{Pre-conditions:}
	\begin{itemize}
		\item User must be logged in.
		\item User must have the correct authorization to make use of this function.
		\item Visitor or Staff must exist.
	\end{itemize}

%//I came up with these two. you guys agree with them?? 
\subsection{Generate Custom Reports}
\textit{Priority: \textcolor{myGreen}{Nice to have}}

\subsubsection{Description}
The generateCustomReports function allows the user to create a custom report based on a query provided by the user.

\subsubsection{Service Contract}
\includegraphics[width=\linewidth]{Reporting/generateCustomReport_serviceContract}

%	\textbf{Paramters:}
%	\begin{itemize}
%		\item Query string - 
%	\end{itemize}
\subsubsection{Pre/Post Conditions}
	\begin{itemize}
		\item User must be logged in.
		\item User must have the correct authorization to make use of this function.
		\item Query must be valid.
	\end{itemize}


\subsection{Export Report To Specific Format}
\textit{Priority: \textcolor{myGreen}{Nice to have}}

\subsubsection{Description}
The exportReportToSpecificFormat function will allow the user to export a report to a specific format which will be specified by the user.

%	\textbf{Parameters required:}
%	\begin{itemize}
%		\item Format - specifies the format to export the report to.
%		\item Report - specifies which report to export to the specific format.
%	\end{itemize}

\subsubsection{Service Contract}
\includegraphics[width=\linewidth]{Reporting/exportToSpecificFormat_serviceContract}

	\textbf{Pre-conditions:}
	\begin{itemize}
		\item The format specified by the user is supported by the system.
		\item User must be logged in.
	\end{itemize}

	
	\newpage
	\section{Authentication}
	


\subsection{Authenticate}
\textit{Priority: \textcolor{red}{Critical}} \\

\subsubsection{Description}
The following section will describe functionality around logging staff members in and out of the web-portal. Note: The use case, process specification and data structure requirements posted in Authenticate encapsulates all use cases under the authentication functionality. \\
\includegraphics[width=\linewidth]{Authentication/DataStructures.jpg}


\subsubsection{Pre-/Post-Conditions}
	\textbf{Pre-conditions:} 
	\begin{itemize}
		\item A user must have an EMPLID from the University of Pretoria 
		\item A user must have an associated password for their EMPLID from the University
		\item A successful connection to LDAP is important
		\item A user must be registered as a staff member on LDAP
		\item A successful validation response after an LDAP authentication is needed to authenticate a user
	\end{itemize}
	\textbf{Post-conditions:} 
		\begin{itemize}
			\item A user is successfully authenticated on the server
		\end{itemize}

\subsubsection{Functionality}
	\includegraphics[width=\linewidth]{Authentication/UseCase.jpg}	
	
\subsubsection{Process Specification}
	\includegraphics[width=\linewidth]{Authentication/Process.jpg}

\subsection{Log In}	
\textit{Priority: \textcolor{red}{Critical}} \\
\subsubsection{Description}
A user must be logged into the system once they have been authenticated. A user is logged in by creating a cookie containing the session ID. \\

\subsubsection{Pre-/Post-Conditions}
	\textbf{Pre-conditions:} 
	\begin{itemize}
		\item A user must have been successfully authenticated by the system to be logged in.
	\end{itemize}
	\textbf{Post-conditions:} 
	\begin{itemize}
		\item A user is successfully logged in and can thus access features which require authentication.
		\item A user is taken to the booking management page on the website
	\end{itemize}
	
\subsection{Log Out}	
\textit{Priority: \textcolor{red}{Critical}} \\
\subsubsection{Description}
The system must be able to log a user out. A user is logged out by destroying the cookie containing their session ID.\\

\subsubsection{Pre-/Post-Conditions}
	\textbf{Pre-conditions:} 
	\begin{itemize}
		\item The user must be logged in.
	\end{itemize}
	\textbf{Post-conditions:} 
	\begin{itemize}
		\item A user is successfully logged in and can thus access features which require authentication.
		\item A user is taken to the booking management page on the website
	\end{itemize}


	%etc....

	

	
\end{document}
